\documentclass{article}
\usepackage{listings}
\lstset{language=Matlab,frame=shadowbox,
	rulesepcolor=\color{red!20!green!20!blue!20},
	keywordstyle=\color{blue!90}\bfseries,
	showstringspaces=false,
	numbers=left,
	numberstyle=\tiny,
	stringstyle=\ttfamily,
	breaklines=true,
	extendedchars=false,}%代码语言使用的是matlab
\usepackage{xcolor}
\usepackage{color}
\definecolor{dkgreen}{rgb}{0,0.6,0}
\usepackage{ctex}
\usepackage{graphicx}
\usepackage[a4paper, body={18cm,22cm}]{geometry}
\usepackage{amsmath,amssymb,amstext,wasysym,enumerate,graphicx}
\usepackage{float,abstract,booktabs,indentfirst,amsmath}
\usepackage{amsthm, bm, hyperref, mathrsfs, tikz, color, framed}
\usepackage{mathrsfs}
\usepackage{array}
\usepackage{booktabs} %调整表格线与上下内容的间隔
\usepackage{multirow}
\usepackage{diagbox}
\renewcommand\arraystretch{1.4}
\usepackage{indentfirst}
\setlength{\parindent}{2em}

\geometry{left=2.8cm,right=2.2cm,top=2.5cm,bottom=2.5cm}
%\geometry{left=3.18cm,right=3.18cm,top=2.54cm,bottom=2.54cm}

\graphicspath{{figures/}}

\title{\heiti 《数值逼近实验报告》 }
\newcounter{theoremname}
\newenvironment{theorem}{\begin{shaded}\stepcounter{theoremname}\par\noindent\textbf{定理\arabic{theoremname}. }}{\end{shaded}\par}
\definecolor{shadecolor}{RGB}{241, 241, 255}
\newcounter{definitionname}
\newenvironment{definition}{\begin{shaded}\stepcounter{definitionname}\par\noindent\textbf{定义\arabic{definitionname}. }}{\end{shaded}\par}
\newcounter{problemname}
\newenvironment{problem}{\begin{shaded}\stepcounter{problemname}\par\noindent\textbf{题目\arabic{problemname}. }}{\end{shaded}\par}
\newenvironment{solution}{\par\noindent\textbf{解答. }}{\par}
\newenvironment{note}{\par\noindent\textbf{注记. }}{\par}

\begin{document}
	
	\maketitle
	
	\vspace{5cm}
	
	
	\begin{table}[h]
		\centering
		\begin{Large}
			\begin{tabular}{p{3cm} p{7cm}<{\centering}}
				学  \qquad  校: &  华中科技大学     \\ \cline{2-2}
				学 \qquad 院:      & 数学与统计学院   \\ \cline{2-2}
				成  \qquad  员: & 何嘉怡 \\ \cline{2-2}
				学\qquad 号: &U201916521 \\ \cline{2-2}
				专\qquad 业:&信息与计算数学\\ \cline{2-2}
				指导教师:       &  \\ \cline{2-2}
			\end{tabular}
		\end{Large}		
	\end{table}
	
	\begin{table}[h]
		
		\setlength{\tabcolsep}{3pt}
		\begin{tabular}{p{2cm} p{5cm}<{\centering}}
			评\qquad 分:&  \\ \cline{2-1}
		\end{tabular}
	\end{table}
	\newpage
	
	\tableofcontents
	\newpage
	
	\section{实验目的}
	我们知道第一类切比雪夫零点是具有最优意义的插值节点,而切比雪夫极值点和勒让德多项式的零点以及极值点是接近最优意义的插值节点,因而这些节点都是非常理想的插值节点。下面我们利用Matlab来编写程序,使用重心公式、牛顿形式或者拉格朗日形式计算各类节点的插值多项式,观察他们的逼近情况。
	研究$f(x)=\left |x \right |+x^2+\sin(10x)$的多项式逼近.
	\section{实验原理}
	数值逼近的知识点
	\section{实验过程}
	\subsection{利用重心公式求切比雪夫零点的插值多项式}
	代码如下:
	\begin{lstlisting}
		function v=interp(f,n,u)
		
		xx=-1:2/u:1;
		x=cos((1:2:(2*n+1))'*pi/(2*n));
		y =f(x);
		w = sin((1:2:(2*n+1))'*pi/(2*n+2)).*(-1).^((0:n)');
		
		numer=zeros(size(u));
		denom=zeros(size(u));
		exact=zeros(size(u));
		
		for j=1:n+1
		del=xx-x(j);
		temp=w(j)./del;
		numer=numer+temp*y(j);
		denom=denom+temp;
		exact(del==0)=j;
		end
		
		v=numer ./denom;
		jj=find(exact);
		v(jj)=y(exact(jj));
		
		plot(xx,v);
	\end{lstlisting}
	\subsection {利用重心形式求勒让德零点的插值多项式}
	\begin{description}
		\item[step1] 首先我们求出n次勒让德多项式的零点
		代码如下:\\
		子程序
		\begin{lstlisting}
			function x=jp(N,alpha,beta)
			
			n=1:N;
			
			a(1)=(alpha+beta+2)/2;
			
			b(1)=(beta-alpha)/2;
			
			a([2:N+1])=(2*n+alpha+beta+1).*(2*n+alpha+beta+2)./(2*(n+1).*(n+alpha+beta+1));
			
			b([2:N+1])=(alpha*alpha-beta*beta)*(2*n+alpha+beta+1)./(2*(n+1).*(n+alpha+beta+1).*(2*n+alpha+beta));
			
			c=(n+alpha).*(n+beta).*(2*n+alpha+beta+2)./((n+1).*(n+alpha+beta+1).*(2*n+alpha+beta));
			
			A=diag(b./a)+diag(1./a([1:N]),1)+diag(c./a([2:N+1]),-1);
			
			x=sort(eig(A))';
		\end{lstlisting}
		主程序
		\begin{lstlisting}
			function x=djp(N,alpha,beta,m)
			
			N1=N-m;
			
			alpha1=alpha+m;
			
			beta1=beta+m;
			
			x=jp(N1,alpha1,beta1);
		\end{lstlisting}
		\item[step2] 下面我们是利用重心公式求解勒让德零点的插值多项式,代码如下:
		\begin{lstlisting}
			function v=Le_interp(f,n,u)
			xx=-1:2/u:1;
			x=djp(n,0,0,0);
			y =f(x);
			w=ones(n+1);
			
			for j=1:n+1
			for i=1:n+1
			if i~=j
			w(j)=w(j)/(x(j)-x(i));
			end
			end
			end
			
			numer=zeros(size(u));
			denom=zeros(size(u));
			exact=zeros(size(u));
			
			for j=1:n+1
			del=xx-x(j);
			temp=w(j)./del;
			numer=numer+temp*y(j);
			denom=denom+temp;
			exact(del==0)=j;
			end
			
			v=numer ./denom;
			jj=find(exact);
			v(jj)=y(exact(jj));
			plot(xx,v);
		\end{lstlisting}
	\end{description}
	\subsection{利用重心公式求勒让德—洛巴特点的插值多项式}
	\begin{lstlisting}
		function v=Le2_interp(f,n,u)
		xx=-1:2/u:1;
		q=djp(n-1,0,0,1);
		x=[-1;q';1];
		y =f(x);
		w=ones(n+1);
		
		for j=1:n+1
		for i=1:n+1
		if i~=j
		w(j)=w(j)/(x(j)-x(i));
		end
		end
		end
		
		numer=zeros(size(u));
		denom=zeros(size(u));
		exact=zeros(size(u));
		
		for j=1:n+1
		del=xx-x(j);
		temp=w(j)./del;
		numer=numer+temp*y(j);
		denom=denom+temp;
		exact(del==0)=j;
		end
		
		v=numer ./denom;
		jj=find(exact);
		v(jj)=y(exact(jj));
		plot(xx,v);
		plot(xx,f(xx));
	\end{lstlisting} 
	
	\subsection{利用重心公式求埃米尔特插值多项式}
	由于切比雪夫零点是最优零点,所以这里我们一致用切比雪夫零点作为节点.\\
	\begin{lstlisting}
		function v=hermite_interp(x,y,dy,n,u)
		m=length(u);
		w=ones(n+1,1);
		
		for j=1:n+1
		for k=1:n+1
		if k~=j
		w(j)=w(j)/(x(j)-x(k));
		end
		end
		end
		
		A=w.^2;
		v=zeros(m,1);
		d=zeros(n+1,1);
		
		for j=1:n+1
		for k=1:n+1
		if k~=j
		d(j)=d(j)+1/(x(j)-x(k));
		end
		end
		end
		
		for j=1:n+1
		B(j)=-2*A(j)*d(j);
		end
		
		temp1=zeros(size(u));
		temp2=zeros(size(u));
		temp3=zeros(size(u));
		exact=zeros(size(u));
		
		for j=1:n+1
		del=u-x(j);
		dell=del.^2;
		
		B(j)=-2*A(j)*d(j);
		temp1=temp1+y(j)*(A(j)./dell+B(j)./del);
		temp3=temp3+A(j)./dell+B(j)./del;
		exact(del==0)=j;
		end
		
		v=(temp1+temp2)./temp3;
		jj=find(exact);
		v(jj)=y(exact(jj));
	\end{lstlisting}
	\section{实验结果}
	\subsection{节点为切比雪夫零点的插值多项式逼近}
	在命令窗口输入:
	\begin{lstlisting}
		f=@(t) abs(t)+t.^2+sin(10*t);
		>> v=Le2_interp(f,n,u)
	\end{lstlisting}
	结果为:
	\begin{figure}[htbp]
		\includegraphics[width=0.3\textwidth]{1}
		\includegraphics[width=0.3\textwidth]{2}
		\includegraphics[width=0.3\textwidth]{3}
	\end{figure}
	以上分别是n=30,n=10,n=7时候的情况.
	\subsection{节点为勒让德零点的插值多项式逼近}
	在命令窗口输入:
	\begin{lstlisting}
		f=@(t) abs(t)+t.^2+sin(10*t);
		>> v=Le_interp(f,n,u)
	\end{lstlisting}
	结果为:
	\begin{figure}[htbp]
		\includegraphics[width=0.3\textwidth]{4}
		\includegraphics[width=0.3\textwidth]{5}
		\includegraphics[width=0.3\textwidth]{6}
	\end{figure}
	\subsection{节点为勒让德—洛巴特点的插值多项式逼近}
	在命令窗口输入:
	\begin{lstlisting}
		f=@(t) abs(t)+t.^2+sin(10*t);
		>> v=Le_interp(f,n,u)
	\end{lstlisting}
	结果为:
	\begin{figure}[htbp]
		\includegraphics[width=0.3\textwidth]{7}
		\includegraphics[width=0.3\textwidth]{8}
		\includegraphics[width=0.3\textwidth]{9}
	\end{figure}
	\subsection{利用重心公式求埃米尔特插值多项式}
	\begin{lstlisting}
		n=10;x=cos((0:n)'*pi/n);
		y =sin(x);
		dy=cos(x);
		u=linspace(-1,1,1000)';
		v=hermite_interp(x,y,dy,n,u);
		plot(x,y,'or',u,v,'b')
	\end{lstlisting}
	\begin{figure}[htbp]
		\includegraphics[width=0.3\textwidth]{10}
		\includegraphics[width=0.3\textwidth]{11}
		\includegraphics[width=0.3\textwidth]{12}
	\end{figure}
	以上分别是n=7,n=10,n=30时的情况.\\
	但是,这里我用的节点是切比雪夫极值点,最开始用的是切比雪夫零点,但是我发现每次图像都非常奇怪,不知道问题出在哪里.
	\section{实验问题}
	
	
	
\end{document}